\documentclass{beamer}
\usepackage[utf8]{inputenc}
\usepackage{listings}
\usepackage{graphicx}
\usepackage{tikz}
\usepackage{algorithmicx}
\usepackage[noend]{algpseudocode}
\usepackage{amsmath}

\usetheme{default}
\setbeamercovered{transparent}

\title{Genetic Algorithms}
\author{Daniel Hawkins}
\date{\today}

\lstset{
  basicstyle=\tiny\ttfamily,
  breaklines=true,
  frame=single
}

\begin{document}

\begin{frame}
\tikz[overlay, remember picture] \node[anchor=south, opacity=0.1] at (current page.south) {\includegraphics[width=\paperwidth]{images/dna_binary.jpeg}};
\titlepage
\end{frame}

\section{Introduction}

\begin{frame}
\frametitle{Genetic Algorithm: Overview}
\begin{itemize}
\item Formalized and popularized in the 1960s-70s by John Holland, inspired by natural selection.
\item Combines Darwin's "survival of the fittest" with selection, crossover, and mutation to solve optimization and search problems.
\item Used in diverse fields like finance, game design, robotics, and drug discovery.
\item Highly adaptable to different types of problems and data.
\item Starts with a population of solutions, iterating and improving over generations.
\item Ability to search large solution spaces, avoiding local optima.
\end{itemize}
\end{frame}

\begin{frame}{Genetic Algorithm Workflow}
  \centering
  \includegraphics[width=0.9\linewidth]{images/flow_diagram.png}
  \end{frame}

\begin{frame}{Genetic Algorithm Workflow}
  \centering
  \includegraphics[width=0.8\linewidth]{images/flow_chart_with_bit_list_maximization.png}
  \end{frame}

\section{Pseudocode}

\begin{frame}[fragile]
\frametitle{Initialize Population}
\centering
\includegraphics[width=.4\linewidth]{images/chromosomes.jpeg}
\footnotesize
\begin{verbatim}
function initPopulation(popSize, chromLen)
    population = []
    for i = 1 to popSize
        chrom = []
        for j = 1 to chromLen
            bit = randomChoice([0, 1])
            chrom = chrom + [bit]
        population = population + [chrom]
    return population
\end{verbatim}
Runtime: O(popSize * chromLen)
\end{frame}

\begin{frame}[fragile]
\frametitle{Calculate Fitness}
\centering
\includegraphics[width=.7\linewidth]{images/fitness.jpeg}
\begin{verbatim}
function fitness(chrom)
    return sum(chrom)
\end{verbatim}
Run Time: O(chromLen)
\end{frame}

\begin{frame}[fragile]
\frametitle{Parent Selection}
\centering
\includegraphics[width=.3\linewidth]{images/selection.jpg}
\small

\begin{lstlisting}[mathescape=true, basicstyle=\ttfamily, frame=none]
function selectParents(population)
    if length(population) == 1
        return population[0], population[0]
    elif length(population) == 2
        return population[0], population[1]
    parent1 = randomChoice(population[0:$\lfloor$length(population)/2$\rfloor$])
    parent2 = randomChoice(population[0:$\lfloor$length(population)/2$\rfloor$])
    return parent1, parent2
\end{lstlisting}
Run Time: O(1)
\end{frame}

\begin{frame}[fragile]
\frametitle{Crossover}
\centering
\includegraphics[width=.7\linewidth]{images/crossover.jpeg}
\begin{verbatim}
function crossover(parent1, parent2)
    point = randomInt(1, length(parent1) - 1)
    child1 = sublist(parent1, 0, point) + 
             sublist(parent2, point, length(parent2))
    child2 = sublist(parent2, 0, point) + 
             sublist(parent1, point, length(parent1))
    return child1, child2
\end{verbatim}
Run Time: O(chromLen)
\end{frame}

\begin{frame}[fragile]
\frametitle{Mutation}
\centering
\includegraphics[width=.7\linewidth]{images/mutate.jpeg}
\small
\begin{verbatim}
function mutate(chrom, mutRate)
    newChrom = []
    for i = 1 to length(chrom)
        if randomFloat(0, 1) > mutRate
            newChrom = newChrom + [chrom[i]]
        else
            if chrom[i] == 0
                newChrom = newChrom + [1]
            else
                newChrom = newChrom + [0]
\end{verbatim}
Run Time: O(chromLen)
\end{frame}

\begin{frame}[fragile]
  \frametitle{Main Algorithm}
  \small
  \begin{verbatim}
  function geneticAlgo(numGen, popSize, chromLen, mutRate)
      population = initPopulation(popSize, chromLen)
      for i = 1 to numGen
          population = sort(population, key=fitness)
          if fitness(population[0]) == chromLen
            return population[0], i + 1
          newPop = []
          while length(newPop) < popSize
              parent1, parent2 = selectParents(population)
              child1, child2 = crossover(parent1, parent2)
              child1 = mutate(child1, mutRate)
              child2 = mutate(child2, mutRate)
              newPop = newPop + [child1, child2]
          population = newPop
      return max(population, key=fitness)
  \end{verbatim}
  Run Time: O(numGen*popSize*chromLen+numGen*popSize*log(popSize))
  \end{frame}

\section{Conclusion}

\begin{frame}
\frametitle{Conclusion}
\includegraphics[width=\linewidth]{images/evolution.jpeg}
\centering
What would you like to optimize through evolution?
\end{frame}

\end{document}
